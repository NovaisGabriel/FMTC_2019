It is dificcult for researchers to obtain real datasets from insurance companies to assess the performance of those metamodeling techiniques.As a result, most of the papers on variable annuity portfolio valuation use synthetic datasets to test the performance of the proposed metamodeling techiniques. Gan and A.Valdez (2017) creates synthetic datasets to facilitate the development and dissemination of reaserch related to the efficient valuation of large variable annuity portfolios. This synthetic dataset was based on the properties of real portfolios of variable annuities and implement a simple Monte Carlo valuation engine that is used to calculate the fair market values and the Greeks of the guarantees embedded in those synthetic variable annuity contracts.

The major properties typically observed on real portfolios of variables annuities contracts:

\begin{itemize}
\item Different contracts may contain different types of guarantees.
\item The contract holder has the option to allocate the money among multiple investment funds.
\item Real variable annuity contracts are issued at different dates and have different times to maturity.
\end{itemize}

There are several types of guarantees, but to create a synthetic portfolio of variabel annuity contracts, Gan and A.Valdez (2017) considers 19 products shown in table. For the synthetic variable annuity policies, they set the rider fees of individual riders in the range of 0.25\% to 0.75\%. The rider fee of the combined guarantees is set equal to the sum of the fees of the individual gurantees minus 0.20\%.

\renewcommand{\arraystretch}{1.5}
\begin{center}
\begin{tabular}{l l l}
		\topline
		\headcol Product & Description                 & Rider Fee  \\
		\midline
		DBRP    & GMDB with return of premium & 0.25\%     \\
		DBRU    & GMDB with annual roll-up     & 0.35\%     \\
		DBSU    & GMDB with annual ratchet     & 0.35\%     \\
		ABRP    & GMAB with return of premium & 0.50\%     \\
		ABRU    & GMAB with annual roll-up     & 0.60\%     \\
		ABSU    & GMAB with annual ratchet     & 0.60\%     \\
		IBRP    & GMIB with return of premium & 0.60\%     \\
		IBRU    & GMIB with annual roll-up     & 0.70\%     \\
		IBSU    & GMIB with annual ratchet     & 0.70\%     \\
		MBRP    & GMMB with return of premium & 0.50\%     \\
		MBRU    & GMMB with annual roll-up     & 0.60\%     \\
		MBSU    & GMMB with annual ratchet     & 0.60\%     \\
		WBRP    & GMWB with return of premium & 0.65\%     \\
		WBRU    & GMWB with annual roll-up     & 0.75\%     \\
		WBSU    & GMWB with annual ratchet     & 0.75\%     \\
		DBAB    & GMDB + GMAB with annual ratchet & 0.75\%     \\
		DBIB    & GMDB + GMIB wwith annual ratchet     & 0.85\%     \\
		DBMB    & GMDB + GMMB with annual ratchet     & 0.75\%     \\
		DBWB    & GMDB + GMWB with annual ratchet & 0.90\% \\
		\bottomlinec
\end{tabular}
\end{center}

The policyholder is allowed to select the investment funds. In dynamic hedging, a fund mapping is used to map an investment fund to a combination of tradable and liquid indices suach as the S\&P500 index. In the synthetic portfolio, account values of the investment funds of a policy were generated randomly from a specified range. The total account values are allocated to the investments funds equally. 

In practice, variable annuity policies in a portfolio are issued at different dates. To value the policies at the valuation date, the policies are aged from the issues dates to the valuation date. The parameters used to generate others variables, as time to maturity and age come from the raw variables displayed in table below.

\renewcommand{\arraystretch}{1.5}
\begin{center}
\begin{tabular}{l l l}
		\topline
		\headcol Feature & Value \\
		\midline
		Policyholder birth date & [1/1/1950,1/1/1980] \\
		Issue date & [1/1/2000,1/1/2014] \\
		Valuation date & 1/6/2014 \\
		Maturity & [15,30]years \\
		Initial account value & [50000,500000] \\
		Female percent & 40\% \\
		Fund fee & 30, 50, 60, 80, 10, 38, 45, 55, 57, 46 bps for Funds 1 to 10 \\
		M\&E fee & 200 bps \\
		\bottomlinec
\end{tabular}
\end{center}

There are two types of scenarios: risk-neutral and real-word. Each of these two scenarios are generated by each measures respectively. Risk-neutral scenarios are used to calculate the fair market values of financial derivatives such as the guarantees embedded in variable annuities. Real-world scenarios are used to calculate solvency capitals or evaluate hedging strategies.

The description of the policy fields at the synthetic dataset is shown below. Gan and A.Valdez (2017), generated 10,000 synthetic variable annuity policies for each of the guarantees types described before. This synthetic portfolio contains 190,000 policies. There are 45 fields in total, including 10 fund values, 10 fund numbers and 10 fund fees.

\renewcommand{\arraystretch}{1.5}
\begin{center}
			\begin{tabular}{l l l}
			\topline
			\headcol Field & Description \\
			\midline
			recordID & Unique identifier of the policy \\
			survivorShip & Positive weighting number\\
			gender & Gender of the policyholder \\
			productType & Product type \\
			issueDate & Issue Date \\
			matDate & Maturity date\\
			birthDate & Birth date of the policyholder \\
			currentDate & Current date\\
			baseFee & M\&E (Mortality \& Expense) fee \\
			riderFee & Rider fee \\
			rollUprate & Roll-up rate\\
			rollUprate & Guaranteed benefit\\
			rollUprate & GMWB balance\\
			wbWithdrawalRate & Guaranteed withdrawal rate\\
			withdrawal & Withdrawal so far\\
			FundValuei & Fund value of the ith investment fund\\
			FundNumi & Fund number of the ith investment fund\\
			FundFeei & Fund management fee of the ith investment fund\\
			\bottomlinec
\end{tabular}
\end{center}    


 